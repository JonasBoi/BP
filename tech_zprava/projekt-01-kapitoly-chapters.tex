%=========================================================================
%===============================================================================

\chapter{Úvod}
Odpovídání na otázky je dobře prozkoumaný a populární úkol z oblasti zpracování přirozeného jazyka. Využití kolem nás, v běžném životě, můžeme vidět třeba u vyhledávacích nástrojů, komunikačních agentů nebo hlasových asistentů.\par 
Výzkum se v dané oblasti soustředí na metody strojového učení pro vyznačení odpovědi na otázku v daném textu. Pro trénink neuronové sítě, která by byla schopná spolehlivě vykonávat takový úkol, je zapotřebí velké množství anotovaných dat.\par 
Kromě porozumění textu jsou v oblasti důležité také algoritmy pro zhodnocení relevantnosti dokumentů používané všemi vyhledávacími nástroji a metody zpracování textu, které jsou užitečné pro optimální funkci vyhledávacích algoritmů.\par
Pro češtinu je množství i velikost dostupných datových sad oproti angličtině malé a~je tedy obtížné dosáhnout s ní podobných výsledků, jako u jazyků s mnoha zdroji (angličtina, němčina, francouzština \dots). Obzvlášť v kontextu toho, že pro většinu úkolů na poli zpracování přirozeného jazyka se v posledních letech nejlépe osvědčily velké modely předtrénované na velmi rozsáhlých korpusech textu. Ty pro češtinu prozatím bohužel nejsou dostupné v~dostatečné kvalitě.\par
V této práci se snažím otestovat přístup, který využívá strojový překlad češtiny do angličtiny, aby mohlo být použito jednoho z robustních předtrénovaných modelů pro porozumění textu. Kromě metod pro porozumění textu a extrakci odpovědi v práci rozebírám algoritmy řazení dokumentů dle jejich relevance k otázce. Jako báze znalostí pro systém schopný na otázky odpovídat je použita česká Wikipedie, jejíž články jsou pro nalezení vhodné odpovědi prohledávány.\par
Začal jsem popsáním metod pro zpracování textu v kapitole \ref{text_processing}. Popsány jsou zde metody reprezentace slov a způsoby zpracování textu vhodné pro porozumění textu a pro algoritmy řazení relevantních dokumentů.\par
Kapitola \ref{language_comprehension} se zabývá popisem neuronových sítí pro porozumění textu a extrakci odpovědi. Popsány jsou zde moderní architektury a modely použité v práci. V kapitole \ref{available_datasets} jsou poté popsány datové sady vhodné pro trénink a vyhodnocení výsledného systému.
Následuje kapitola \ref{design_and_implementation}, kde je popsán návrh a implementace systému založeného na zmíněných postupech a kapitola \ref{system_evaluation}, kde je výsledný systém vyhodnocen s použitím standardních metrik.




%=========================================================================
%===============================================================================

\chapter{Reprezentace slov a zpracování textu}
\label{text_processing}

Předtím, než se vrhneme na vysvětlení poměrně složitých jazykových modelů a vyhledávacích algoritmů, bude v této kapitole vysvětleno pár základních pojmů a technik.\par 
Počítač lidské řeči na elementární úrovni příliš nerozumí, jsou mu ale mnohem bližší čísla a statistika. Proto musíme nejdříve porozumět tomu, s jakou reprezentací slov moderní jazykové modely pracují a do jaké podoby je tedy potřeba zdrojový text dostat. Vysvětleny jsou tedy termíny související s reprezentací slov pomocí vektorů, neboli \uv{word embeddings} a tokenizace textu.\par 
Také je potřeba prozkoumat možnosti zpracování textů a slov na jejich morfologické úrovni. Tato úloha je na pomezí informatiky a lingvistiky a používají se pro ni speciální nástroje. Použité techniky jako lemmatizace a odstranění stop-slov zajistí optimální funkčnost některých použitých algoritmů.\par
Po přečtení kapitoly by měl být čtenář teoreticky vybaven pro pochopení konceptů vysvětlených v kapitolách \ref{language_comprehension} a \ref{document_retrieval}, které na zde popsaných technologiích staví a částečně se s nimi překrývají.

%=========================================================================
\section{Reprezentace slov}
Jak už bylo řečeno, je složité pracovat s textem v takové podobě, na jakou jsou lidé zvyklí. Počítač v ní nedokáže vidět důležité sémantické souvislosti, které jsou pro přirozený jazyk tak důležité. Musíme tedy z textu a slov v něm získat nějakou matematickou reprezentaci, která je počítači bližší a se kterou dokáží dále pracovat neuronové sítě. \par
Pro takovou reprezentaci jsou nejpoužívanější vektory pevně dané dimenze známé jako \uv{word embeddings}. Pro představu např. 300 desetinných čísel pro popis každého slova v textu. Vektory slov ve slovníku nesou důležitou informaci o sémantické příbuznosti jednotlivých slov. Po promítnutí do spojitého prostoru jsou si vektory, které reprezentující sémanticky podobná slova, blízko.\par
Po aplikaci těchto myšlenek zmíněných v článcích \cite{mikolov2013embeddings} a \cite{mikolov2013_2} byl zaznamenán výrazný pokrok v úlohách z oblasti zpracování přirozeného jazyka, jako strojový překlad, porozumění textu, odhad emocí, klasifikace atp.
Vektory reprezentující slova jsou většinou získávány analýzou rozsáhlých textových korpusů. Snaží se zachytit sémantickou blízkost slov na základě jejich koexistence v podobných kontextech.\par 
Například slova \uv{pohovka} a \uv{gauč} se nejspíše budou vyskytovat v podobných situacích, protože jsou to synonyma a jejich sémantika je téměř totožná.\par
V následující části budou popsány vlastnosti a nejznámější metody pro získání vektorových reprezentací.

%=========================================================================

\subsection{Word2vec}

Word2vec je technika prezentována v článku \cite{mikolov2013_2}.

%=========================================================================

\subsection{GloVe}

GloVe je technika prezentována v článku \cite{GloVe}

%=========================================================================

\subsection{Wordpiece embeddings}
\blindtext[4]

%=========================================================================
\section{Tokenizace a předzpracování}
\blindtext[2]
\subsection{Tokenizace vstupního textu}
\blindtext[2]
\subsection{Získání lemmat}
\blindtext[1]
\subsection{Odstranění stopslov a převod na malá písmena}
\blindtext[1]



%=========================================================================
%===============================================================================

\chapter{Strojové učení pro extrakci odpovědi}
\label{language_comprehension}

\blindtext[2]

%=========================================================================
\section{Neuronové sítě}
\blindtext[6]

%=========================================================================
\section{Nejlepší architektury pro porozumění textu}
\blindtext[8]

%=========================================================================
\section{Předtrénované modely BERT a ALBERT}
\blindtext[8]



%=========================================================================
%===============================================================================

\chapter{Vyhledávání relevantních dokumentů}
\label{document_retrieval}

\blindtext[2]

%=========================================================================
\section{Algoritmy pro řazení dokumentů}
\blindtext[8]

%=========================================================================
\section{Vyznačení pojmenovaných entit v otázce}
\blindtext[8]

%=========================================================================
\section{Úskalí velké báze dat jako Wikipedie}
\blindtext[4]




%=========================================================================
%===============================================================================

\chapter{Dostupné datové sady a výběr}
\label{available_datasets}

\blindtext[2]

%=========================================================================
\section{Úskalí dostupných zdrojů pro češtinu oproti angličtině}
\blindtext[6]

%=========================================================================
\section{Datové sady pro češtinu}
\blindtext[4]

%=========================================================================
\section{Datové sady pro angličtinu}
\blindtext[4]

%=========================================================================
\section{Výběr trénovacích a testovacích dat}
\blindtext[4]




%=========================================================================
%===============================================================================

\chapter{Návrh systému a implementace jednotlivých komponent}
\label{design_and_implementation}

\blindtext[2]

%=========================================================================
\section{Zvolený přístup k problému}
\blindtext[6]

%=========================================================================
\section{Návrh jednotlivých částí systému}
\blindtext[15]

%=========================================================================
\section{Použité nástroje a technologie pro implementaci}
\blindtext[6]

%=========================================================================
\section{Implementace a trénink readeru}
\blindtext[8]

%=========================================================================
\section{Implementace retrieveru}
\blindtext[10]

%=========================================================================
\section{Popis výsledného systému}
\blindtext[8]




%=========================================================================
%===============================================================================

\chapter{Vyhodnocení systému a rozbor chyb}
\label{system_evaluation}

\blindtext[2]

%=========================================================================
\section{Vysvětlení základních metrik}
\blindtext[4]

%=========================================================================
\section{Postup při vyhodnocování výsledného systému}
\blindtext[3]

%=========================================================================
\section{Porovnání výsledků se současným stavem poznání}
\blindtext[8]

%=========================================================================
\section{Rozbor chyb a možnosti dalšího vývoje}
\blindtext[5]




%=========================================================================
%===============================================================================

\chapter{Závěr}
\label{conclusion}
\blindtext[4]

%=========================================================================
%===============================================================================
